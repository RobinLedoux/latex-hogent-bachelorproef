%---------- Inleiding ---------------------------------------------------------

%\paragraph{Opmerking}

\section{Inleiding}%
\label{sec:inleiding}
Financiële instellingen gebruiken synchrone data-mirroring voor twee geografisch nabijgelegen datacenters om de beschikbaarheid van hun mainframe-omgevingen te waarborgen \autocite{bourbonnais2015}. Europese voorschriften stellen strikte eisen aan de operationele veerkracht en continuïteit van financiële instellingen.
De Digital Operational Resilience Act verplicht financiële instellingen om een uitgebreid ICT-risico\-beheer\-kader te implementeren dat de beschikbaarheid en integriteit van kritieke systemen waarborgt, inclusief plannen voor redundantie of herstel wanneer er zich verstoringen voordoen \autocite{clausmeier2023regulation}. 
Het probleem ontstaat echter wanneer beide datacenters tegelijkertijd uitvallen, waardoor de continuïteit van de diensten en de integriteit van transacties in gevaar komt. 
Om dit te vermijden is het noodzakelijk een extra, geografisch gescheiden datacentrum toe te voegen, waarop alle transacties asynchroon kunnen worden gerepliceerd \autocite{snedaker2013business}.
Het doel van deze bachelorproef is het analyseren van de risico's die gepaard gaan met het uitbreiden van een synchrone mainframe opstelling, bestaande uit twee nabijgelegen datacenters, naar een architectuur met een derde, geografisch gescheiden site buiten de maximale afstand voor synchrone data-mirroring. 
Hierbij wordt onderzocht hoe DB2 transactie logreplicatie kan worden ingezet om data-consistentie en herstelbaarheid te waarborgen en de impact op Recovery Point Objective (RPO) en Recovery Time Objective (RTO) inzichtelijk te maken.

%\begin{itemize}
%  \item kaderen thema
%  \item de doelgroep
%  \item de probleemstelling en (centrale) onderzoeksvraag
% \item de onderzoeksdoelstelling
%\end{itemize}
%---------- Stand van zaken ---------------------------------------------------

\section{Literatuurstudie}%
\label{sec:literatuurstudie}
Financiële instellingen hebben nood aan systemen die de beschikbaarheid en integriteit van transacties waarborgen \autocite{Wishartsmith2024}.
IBM biedt hiervoor een oplossing genaamd DB2 PureScale, een oplossing ontworpen voor continue beschikbaarheid \autocite{ibm2025db2purescalelux}. 
DB2 PureScale heeft meerdere toepassingen, waaronder Q Replication / InfoSphere Change Data Capture (CDC) \autocite{ibm2025db2purescaleoptionslux}. 
CDC laat toe om sites met een hoge gegevensdoorvoer weinig vertragingstijd te bezorgen ondanks afstanden van duizenden kilometers tussen sites \autocite{ibm2025db2purescaleoptionslux}.
DB2 PureScale is echter ontworpen voor DB2 op Linux, Windows en Unix en is niet beschikbaar op DB2 voor het Z/OS-besturingsssysteem \autocite{ibm2025db2}.
IBM mainframes met het Z/OS-besturingsssysteem maken daarom gebruik van DB2 data sharing \autocite{ibm2025db2}.
Data sharing ondersteunt synchrone replicatie tussen de nodes van een data-sharing group, maar de prestaties zijn afhankelijk van de netwerkinfrastructuur \autocite{bari2008}. 
Om de vertragingstijd bij het synchroniseren te minimaliseren is er tussen de sites nood aan dark fibre connecties \autocite{bari2008}.
Synchrone DB2 data sharing vereist dat transacties pas als voltooid worden beschouwd wanneer ze op alle betrokken nodes zijn vastgelegd \autocite{bari2008}. 
De maximale afstand tussen sites is beperkt door de vertragingstijd van de verbindingen, die immers ook te maken heeft met de fysieke beperkingen van de lichtsnelheid in glasvezel \autocite{bari2008}. 
Door deze beperkingen kan een derde geografisch afgelegen site geen synchrone DB2 data sharing gebruiken.
Bovendien moet de afstand van de derde site ver genoeg zijn zodat bij natuurverschijnselen zoals een aardbeving de data niet permanent vernietigd kan worden \autocite{ibm2023disaster}.
Daarom zijn er voor asynchrone replicatie twee methodes ontworpen voor DB2 op Z/OS, met name GDPS Global Mirror en Q Replication \autocite{bourbonnais2015}.
Met GDPS Global Mirror kan het volgens \textcite{bourbonnais2015} een uur duren tot de data hersteld is omdat de replicatie gebeurt op opslagniveau, niet op databank niveau.
Q Replication repliceert daarentegen enkel de relationele tabellen, niet de volledige disk volumes waardoor de hersteltijd volgens \textcite{bourbonnais2015} maximaal een minuut bedraagt.
Echter is er nog het intrinsieke probleem dat asynchrone replicatie met zich meebrengt: wijzigingen worden niet tegelijkertijd met de primaire transacties bevestigd.
Hierdoor kan de data op de derde site achterlopen en kunnen er nog onvoltooide transacties in het systeem aanwezig zijn. Bij failover is er dan aanvullende verwerking nodig om de database naar een consistente staat te brengen.
Asynchrone replicatie biedt dus uiteindelijke data-consistentie, maar vereist aanzienlijke verwerking om de afwijkende toestand na een storing te herstellen waardoor de totale hersteltijd toeneemt en zowel RPO als RTO beïnvloed worden \autocite{coulouris2005distributed}.


% Voor literatuurverwijzingen zijn er twee belangrijke commando's:
% \autocite{KEY} => (Auteur, jaartal) Gebruik dit als de naam van de auteur
%   geen onderdeel is van de zin.
% \textcite{KEY} => Auteur (jaartal)  Gebruik dit als de auteursnaam wel een
%   functie heeft in de zin (bv. ``Uit onderzoek door Doll & Hill (1954) bleek
%   ...'')

%---------- Methodologie ------------------------------------------------------
\section{Methodologie}%
\label{sec:methodologie}
Het onderzoek voor deze bachelorproef zal worden uitgevoerd met behulp van een technische literatuurstudie, een architecturale analyse en een risicoanalyse. 
In de eerste fase wordt een literatuurstudie uitgevoerd op basis van academische publicaties en technische documentatie.
Hierbij worden de verschillende manieren van DB2 logreplicatie naast elkaar gelegd en hun verschillende eigenschappen genoteerd met transactionele consistentie en herstelmechanismen als centraal doel.
Voor deze fase wordt gebruik gemaakt van IBM Redbooks en officiële IBM-documentatie voor:
\begin{itemize}
    \item DB2 voor Z/OS
    \item DB2 data sharing
    \item GDPS Global Mirror
    \item Q Replication
\end{itemize}
Aanvullend wordt ook gekeken naar academische literatuur over:
\begin{itemize}
    \item gedistribueerde systemen
    \item synchrone tegenover asynchrone replicatie
\end{itemize}
Uit deze fase komt een gestructureerd overzicht van synchrone tegenover asynchrone replicatietechnieken in DB2 met eigenschappen die betrekking hebben tot consistentie, vertragingstijd en herstel.
Componenten zoals coupling facilities, log streams en netwerkverbindingen komen hierbij aan bod.
Er wordt ook gekaderd wat RPO en RTO inhoudt binnen een mainframe context.
Geschatte tijd voor deze fase bedraagt 3 weken.

In de tweede fase wordt een vergelijkende architecturale analyse uitgevoerd van verschillende opstellingen om data integriteit te behouden bij mainframe opstellingen.
Concreet gaat dit om het vergelijken van een gesynchroniseerde mainframe opstelling met twee sites ten opzichte van deze gesynchroniseerde opstelling samen met een asynchrone derde site.
Hierbij wordt specifiek gekeken wat de gevolgen zijn voor de dataconsistentie, hoe bij elke opstelling een herstelprocedure wordt opgestart bij een uitvalsprocedure en wat de gevolgen zijn voor RPO en RTO\@.
Met vergelijkende tabellen worden dataconsistentie, herstelmechanismen en de implicaties voor RPO en RTO gevisualiseerd.
Er wordt aangetoond wat de bestaande maatregelen zijn die genomen worden om dataconsistentie bij herstelscenario's te behouden.

Geschatte tijd voor deze fase bedraagt 2 weken.
    
In de derde fase wordt een risicoanalyse waarbij wordt onderzocht welke bijkomende risico's asynchrone replicatie introduceert in een bancaire context.
Hier wordt specifiek gekeken naar wat mogelijke scenario's kunnen zijn waarin een herstelprocedure zal moeten worden opgestart in de DB2 omgeving.
Dit resulteert in een concrete lijst met faalscenario's.
Geschatte tijd voor deze fase bedraagt 3 weken.

In een vierde fase wordt geanalyseerd wat de mogelijke zwakheden zijn van DB2 voor Z/OS bij de verschillende scenario's beschreven in de vorige stap.
Er wordt gekeken naar asynchrone replactie van commits, incomplete Logical Units of Works (LUW), de volgorde van log replicatie en de gevolgen van handmatige herstelprocedures.
Elk replicatiemechanisme zal resulteren in een matrix waarin de zwakheden zichtbaar worden gesteld.
Geschatte tijd voor deze fase: 3 weken.

In de vijfde fase wordt een analytische inschatting gedaan op de waarschijnlijkheid en impact van RPO, RTO en dataconsistentie gebaseerd op synchrone en asynchrone architecturen.
Op basis van deze inschatting wordt een risicomatrix gemaakt.
Geschatte tijd: 2 weken.

In de zesde fase worden de risico's bepaald op basis van de waarschijnlijkheid en de impact die synchrone en asynchrone oplossingen met zich meebrengen.
Deze fase resulteert in een gerangschikte lijst van risico's.
Geschatte tijd: 2 weken.

In de zevende fase wordt gekeken hoe de risico's kunnen worden geminimaliseerd door de onderzochte technieken.
Dit omvat gecontroleerd inhalen van DB2 transactie logs, hoe een asynchrone derde site kan worden ingezet en het resultaat op RPO bij de derde site.
Hieruit komt een tabel met aanbevelingen per risico.
Geschatte tijd: 3 weken.

%---------- Verwachte resultaten ----------------------------------------------
\section{Verwacht resultaat, conclusie}%
\label{sec:verwachte_resultaten}

Hier beschrijf je welke resultaten je verwacht. Als je metingen en simulaties uitvoert, kan je hier al mock-ups maken van de grafieken samen met de verwachte conclusies. Benoem zeker al je assen en de onderdelen van de grafiek die je gaat gebruiken. Dit zorgt ervoor dat je concreet weet welk soort data je moet verzamelen en hoe je die moet meten.

Wat heeft de doelgroep van je onderzoek aan het resultaat? Op welke manier zorgt jouw bachelorproef voor een meerwaarde?

Hier beschrijf je wat je verwacht uit je onderzoek, met de motivatie waarom. Het is \textbf{niet} erg indien uit je onderzoek andere resultaten en conclusies vloeien dan dat je hier beschrijft: het is dan juist interessant om te onderzoeken waarom jouw hypothesen niet overeenkomen met de resultaten.

