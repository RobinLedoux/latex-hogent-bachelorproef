%---------- Inleiding ---------------------------------------------------------

%\paragraph{Opmerking}

\section{Inleiding}%
\label{sec:inleiding}

Financiële instellingen gebruiken synchrone data-mirroring voor twee geografisch nabijgelegen datacenters om de beschikbaarheid van hun mainframe-omgevingen te waarborgen \autocite{bourbonnais2015}. Europese voorschriften stellen strikte eisen aan de operationele veerkracht en continuïteit van financiële instellingen.
De Digital Operational Resilience Act verplicht financiële instellingen om een uitgebreid ICT-risico\-beheer\-kader te implementeren dat de beschikbaarheid en integriteit van kritieke systemen waarborgt, inclusief plannen voor redundantie of herstel wanneer er zich verstoringen voordoen \autocite{clausmeier2023regulation}. 
Het probleem ontstaat echter wanneer beide datacenters tegelijkertijd uitvallen, waardoor de continuïteit van de diensten en de integriteit van transacties in gevaar komt. 
Om dit te vermijden is het noodzakelijk een extra, geografisch gescheiden datacentrum toe te voegen, waarop alle transacties asynchroon kunnen worden gerepliceerd \autocite{snedaker2013business}.
Het doel van deze bachelorproef is het analyseren van de risico's die gepaard gaan met het uitbreiden van een synchrone mainframe opstelling, bestaande uit twee nabijgelegen datacenters, naar een architectuur met een derde, geografisch gescheiden site buiten de maximale afstand voor synchrone data-mirroring. 
Hierbij wordt onderzocht hoe DB2 transaction logreplicatie kan worden ingezet om data-consistentie en herstelbaarheid te waarborgen en de impact op RPO en RTO inzichtelijk te maken.

%\begin{itemize}
%  \item kaderen thema
%  \item de doelgroep
%  \item de probleemstelling en (centrale) onderzoeksvraag
% \item de onderzoeksdoelstelling
%\end{itemize}
%---------- Stand van zaken ---------------------------------------------------

\section{Literatuurstudie}%
\label{sec:literatuurstudie}

Hier beschrijf je de \emph{state-of-the-art} rondom je gekozen onderzoeksdomein, d.w.z.\ een inleidende, doorlopende tekst over het onderzoeksdomein van je bachelorproef. Je steunt daarbij heel sterk op de professionele \emph{vakliteratuur}, en niet zozeer op populariserende teksten voor een breed publiek. Wat is de huidige stand van zaken in dit domein, en wat zijn nog eventuele open vragen (die misschien de aanleiding waren tot je onderzoeksvraag!)?

Je mag de titel van deze sectie ook aanpassen (literatuurstudie, stand van zaken, enz.). Zijn er al gelijkaardige onderzoeken gevoerd? Wat concluderen ze? Wat is het verschil met jouw onderzoek?

Verwijs bij elke introductie van een term of bewering over het domein naar de vakliteratuur! Denk zeker goed na welke werken je refereert en waarom.

Draag zorg voor correcte literatuurverwijzingen! Een bronvermelding hoort thuis \emph{binnen} de zin waar je je op die bron baseert, dus niet er buiten! Maak meteen een verwijzing als je gebruik maakt van een bron. Doe dit dus \emph{niet} aan het einde van een lange paragraaf. Baseer nooit teveel aansluitende tekst op eenzelfde bron.

Als je informatie over bronnen verzamelt in JabRef, zorg er dan voor dat alle nodige info aanwezig is om de bron terug te vinden (zoals uitvoerig besproken in de lessen Research Methods).

Financiële instellingen hebben nood aan systemen die de beschikbaarheid en integriteit van transaties waarborgen \autocite{Wishartsmith2024}.
IBM biedt hiervoor een oplossing genaamd DB2 purescale, een oplossing ontworpen voor continue beschikbaarheid \autocite{ibm2025db2purescale}. 
DB2 purescale heeft meerdere toepassingen, waaronder Q replication / InfoSphere Change Data Capture (CDC). 
CDC laat toe om sites met een hoge doorvoer van data weinig vertragingstijd te bezorgen ondanks afstanden van duizenden kilometers tussen sites \autocite{ibm2025db2purescaleoptionslux}.
DB2 purescale is echter ontworpen voor DB2 op Linux, Windows en Unix en is niet beschikbaar op DB2 voor het Z/OS besturingsssysteem.
IBM mainframes die het Z/OS mainframe bevatten, maken daarom gebruik van DB2 data sharing \autocite{bibid}.
Data sharing ondersteunt synchrone replicatie tussen de nodes van een data-sharing group, maar de prestaties zijn afhankelijk van de netwerkinfrastructuur. 
Om de vertragingstijd bij het synchroniseren te minimaliseren is er tussen de sites nood aan dark fibre connecties \autocite{bari2008}. 


% Voor literatuurverwijzingen zijn er twee belangrijke commando's:
% \autocite{KEY} => (Auteur, jaartal) Gebruik dit als de naam van de auteur
%   geen onderdeel is van de zin.
% \textcite{KEY} => Auteur (jaartal)  Gebruik dit als de auteursnaam wel een
%   functie heeft in de zin (bv. ``Uit onderzoek door Doll & Hill (1954) bleek
%   ...'')

%---------- Methodologie ------------------------------------------------------
\section{Methodologie}%
\label{sec:methodologie}

Hier beschrijf je hoe je van plan bent het onderzoek te voeren. Welke onderzoekstechniek ga je toepassen om elk van je onderzoeksvragen te beantwoorden? Gebruik je hiervoor literatuurstudie, interviews met belanghebbenden (bv.~voor requirements-analyse), experimenten, simulaties, vergelijkende studie, risico-analyse, PoC, \ldots?

Valt je onderwerp onder één van de typische soorten bachelorproeven die besproken zijn in de lessen Research Methods (bv.\ vergelijkende studie of risico-analyse)? Zorg er dan ook voor dat we duidelijk de verschillende stappen terug vinden die we verwachten in dit soort onderzoek!

Vermijd onderzoekstechnieken die geen objectieve, meetbare resultaten kunnen opleveren. Enquêtes, bijvoorbeeld, zijn voor een bachelorproef informatica meestal \textbf{niet geschikt}. De antwoorden zijn eerder meningen dan feiten en in de praktijk blijkt het ook bijzonder moeilijk om voldoende respondenten te vinden. Studenten die een enquête willen voeren, hebben meestal ook geen goede definitie van de populatie, waardoor ook niet kan aangetoond worden dat eventuele resultaten representatief zijn.

Uit dit onderdeel moet duidelijk naar voor komen dat je bachelorproef ook technisch voldoen\-de diepgang zal bevatten. Het zou niet kloppen als een bachelorproef informatica ook door bv.\ een student marketing zou kunnen uitgevoerd worden.

Je beschrijft ook al welke tools (hardware, software, diensten, \ldots) je denkt hiervoor te gebruiken of te ontwikkelen.

Probeer ook een tijdschatting te maken. Hoe lang zal je met elke fase van je onderzoek bezig zijn en wat zijn de concrete \emph{deliverables} in elke fase?

%---------- Verwachte resultaten ----------------------------------------------
\section{Verwacht resultaat, conclusie}%
\label{sec:verwachte_resultaten}

Hier beschrijf je welke resultaten je verwacht. Als je metingen en simulaties uitvoert, kan je hier al mock-ups maken van de grafieken samen met de verwachte conclusies. Benoem zeker al je assen en de onderdelen van de grafiek die je gaat gebruiken. Dit zorgt ervoor dat je concreet weet welk soort data je moet verzamelen en hoe je die moet meten.

Wat heeft de doelgroep van je onderzoek aan het resultaat? Op welke manier zorgt jouw bachelorproef voor een meerwaarde?

Hier beschrijf je wat je verwacht uit je onderzoek, met de motivatie waarom. Het is \textbf{niet} erg indien uit je onderzoek andere resultaten en conclusies vloeien dan dat je hier beschrijft: het is dan juist interessant om te onderzoeken waarom jouw hypothesen niet overeenkomen met de resultaten.

