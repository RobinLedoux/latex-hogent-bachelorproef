%---------- Inleiding ---------------------------------------------------------

%\paragraph{Opmerking}

\section{Inleiding}%
\label{sec:inleiding}

Financiële instellingen gebruiken synchrone data-mirroring voor twee geografisch nabijgelegen datacenters om de beschikbaarheid van hun mainframe-omgevingen te waarborgen \autocite{bourbonnais2015}. Europese voorschriften stellen strikte eisen aan de operationele veerkracht en continuïteit van financiële instellingen.
De Digital Operational Resilience Act verplicht financiële instellingen om een uitgebreid ICT-risico\-beheer\-kader te implementeren dat de beschikbaarheid en integriteit van kritieke systemen waarborgt, inclusief plannen voor redundantie of herstel wanneer er zich verstoringen voordoen \autocite{clausmeier2023regulation}. 
Het probleem ontstaat echter wanneer beide datacenters tegelijkertijd uitvallen, waardoor de continuïteit van de diensten en de integriteit van transacties in gevaar komt. 
Om dit te vermijden is het noodzakelijk een extra, geografisch gescheiden datacentrum toe te voegen, waarop alle transacties asynchroon kunnen worden gerepliceerd \autocite{snedaker2013business}.
Het doel van deze bachelorproef is het analyseren van de risico's die gepaard gaan met het uitbreiden van een synchrone mainframe opstelling, bestaande uit twee nabijgelegen datacenters, naar een architectuur met een derde, geografisch gescheiden site buiten de maximale afstand voor synchrone data-mirroring. 
Hierbij wordt onderzocht hoe DB2 transaction logreplicatie kan worden ingezet om data-consistentie en herstelbaarheid te waarborgen en de impact op RPO en RTO inzichtelijk te maken.

%\begin{itemize}
%  \item kaderen thema
%  \item de doelgroep
%  \item de probleemstelling en (centrale) onderzoeksvraag
% \item de onderzoeksdoelstelling
%\end{itemize}
%---------- Stand van zaken ---------------------------------------------------

\section{Literatuurstudie}%
\label{sec:literatuurstudie}

Financiële instellingen hebben nood aan systemen die de beschikbaarheid en integriteit van transacties waarborgen \autocite{Wishartsmith2024}.
IBM biedt hiervoor een oplossing genaamd DB2 PureScale, een oplossing ontworpen voor continue beschikbaarheid \autocite{ibm2025db2purescalelux}. 
DB2 PureScale heeft meerdere toepassingen, waaronder Q Replication / InfoSphere Change Data Capture (CDC) \autocite{ibm2025db2purescaleoptionslux}. 
CDC laat toe om sites met een hoge gegevensdoorvoer weinig vertragingstijd te bezorgen ondanks afstanden van duizenden kilometers tussen sites \autocite{ibm2025db2purescaleoptionslux}.
DB2 PureScale is echter ontworpen voor DB2 op Linux, Windows en Unix en is niet beschikbaar op DB2 voor het Z/OS-besturingsssysteem \autocite{ibm2025db2}.
IBM mainframes met het Z/OS-besturingsssysteem maken daarom gebruik van DB2 data sharing \autocite{ibm2025db2}.
Data sharing ondersteunt synchrone replicatie tussen de nodes van een data-sharing group, maar de prestaties zijn afhankelijk van de netwerkinfrastructuur \autocite{bari2008}. 
Om de vertragingstijd bij het synchroniseren te minimaliseren is er tussen de sites nood aan dark fibre connecties \autocite{bari2008}.
Synchrone DB2 data sharing vereist dat transacties pas als voltooid worden beschouwd wanneer ze op alle betrokken nodes zijn vastgelegd \autocite{bari2008}. 
De maximale afstand tussen sites is beperkt door de vertragingstijd van de verbindingen, die immers ook te maken heeft met de fysieke beperkingen van de lichtsnelheid in glasvezel \autocite{bari2008}. 
Door deze beperkingen kan een derde geografisch afgelegen site geen synchrone DB2 data sharing gebruiken.
Bovendien moet de afstand van de derde site ver genoeg zijn zodat bij natuurverschijnselen zoals een aardbeving de data niet permanent vernietigd kan worden \autocite{ibm2023disaster}.
Daarom zijn er voor asynchrone replicatie twee methodes ontworpen voor DB2 op Z/OS, met name GDPS Global Mirror en Q Replication \autocite{bourbonnais2015}.
Met GDPS Global Mirror kan het volgens \textcite{bourbonnais2015} een uur duren tot de data hersteld is omdat de replicatie gebeurt op opslagniveau, niet op databank niveau.
Q Replication repliceert daarentegen enkel de relationele tabellen, niet de volledige disk volumes waardoor de hersteltijd volgens \textcite{bourbonnais2015} maximaal een minuut bedraagt.
Echter is er nog het intrinsieke probleem dat asynchrone replicatie met zich meebrengt: wijzigingen worden niet tegelijkertijd met de primaire transacties bevestigd.
Hierdoor kan de data op de derde site achterlopen en kunnen er nog onvoltooide transacties in het systeem aanwezig zijn. Bij failover is er dan aanvullende verwerking nodig om de database naar een consistente staat te brengen.
Asynchrone replicatie biedt dus uiteindelijke data-consistentie, maar vereist aanzienlijke verwerking om de afwijkende toestand na een storing te herstellen waardoor de totale hersteltijd toeneemt en zowel RPO als RTO beïnvloed worden \autocite{coulouris2005distributed}.


% Voor literatuurverwijzingen zijn er twee belangrijke commando's:
% \autocite{KEY} => (Auteur, jaartal) Gebruik dit als de naam van de auteur
%   geen onderdeel is van de zin.
% \textcite{KEY} => Auteur (jaartal)  Gebruik dit als de auteursnaam wel een
%   functie heeft in de zin (bv. ``Uit onderzoek door Doll & Hill (1954) bleek
%   ...'')

%---------- Methodologie ------------------------------------------------------
\section{Methodologie}%
\label{sec:methodologie}

Hier beschrijf je hoe je van plan bent het onderzoek te voeren. Welke onderzoekstechniek ga je toepassen om elk van je onderzoeksvragen te beantwoorden? Gebruik je hiervoor literatuurstudie, interviews met belanghebbenden (bv.~voor requirements-analyse), experimenten, simulatieselijkende studie, risico-analyse, PoC, \ldots?

Valt je onderwerp onder één van de typische soorten bachelorproeven die besproken zijn in de lessen Research Methods (bv.\ vergelijkende studie of risico-analyse)? Zorg er dan ook voor dat we duidelijk de verschillende stappen terug vinden die we verwachten in dit soort onderzoek!

Vermijd onderzoekstechnieken die geen objectieve, meetbare resultaten kunnen opleveren. Enquêtes, bijvoorbeeld, zijn voor een bachelorproef informatica meestal \textbf{niet geschikt}. De antwoorden zijn eerder meningen dan feiten en in de praktijk blijkt het ook bijzonder moeilijk om voldoende respondenten te vinden. Studenten die een enquête willen voeren, hebben meestal ook geen goede definitie van de populatie, waardoor ook niet kan aangetoond worden dat eventuele resultaten representatief zijn.

Uit dit onderdeel moet duidelijk naar voor komen dat je bachelorproef ook technisch voldoen\-de diepgang zal bevatten. Het zou niet kloppen als een bachelorproef informatica ook door bv.\ een student marketing zou kunnen uitgevoerd worden.

Je beschrijft ook al welke tools (hardware, software, diensten, \ldots) je denkt hiervoor te gebruiken of te ontwikkelen.

Probeer ook een tijdschatting te maken. Hoe lang zal je met elke fase van je onderzoek bezig zijn en wat zijn de concrete \emph{deliverables} in elke fase?

%---------- Verwachte resultaten ----------------------------------------------
\section{Verwacht resultaat, conclusie}%
\label{sec:verwachte_resultaten}

Hier beschrijf je welke resultaten je verwacht. Als je metingen en simulaties uitvoert, kan je hier al mock-ups maken van de grafieken samen met de verwachte conclusies. Benoem zeker al je assen en de onderdelen van de grafiek die je gaat gebruiken. Dit zorgt ervoor dat je concreet weet welk soort data je moet verzamelen en hoe je die moet meten.

Wat heeft de doelgroep van je onderzoek aan het resultaat? Op welke manier zorgt jouw bachelorproef voor een meerwaarde?

Hier beschrijf je wat je verwacht uit je onderzoek, met de motivatie waarom. Het is \textbf{niet} erg indien uit je onderzoek andere resultaten en conclusies vloeien dan dat je hier beschrijft: het is dan juist interessant om te onderzoeken waarom jouw hypothesen niet overeenkomen met de resultaten.

