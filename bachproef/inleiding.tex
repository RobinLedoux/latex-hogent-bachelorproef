%%=============================================================================
%% Inleiding
%%=============================================================================

\chapter{\IfLanguageName{dutch}{Inleiding}{Introduction}}%
\label{ch:inleiding}

Financiële instellingen gebruiken synchrone data-mirroring voor twee geografisch nabijgelegen datacenters om de beschikbaarheid van hun mainframe-omgevingen te waarborgen \autocite{bourbonnais2015}. 
Europese voorschriften stellen strikte eisen aan de operationele veerkracht en continuïteit van financiële instellingen.
De Digital Operational Resilience Act verplicht financiële instellingen om een uitgebreid ICT-risico\-beheer\-kader te implementeren dat de beschikbaarheid en integriteit van kritieke systemen waarborgt, inclusief plannen voor redundantie of herstel wanneer er zich verstoringen voordoen \autocite{clausmeier2023regulation}. 

\section{\IfLanguageName{dutch}{Probleemstelling}{Problem Statement}}%
\label{sec:probleemstelling}

Het probleem ontstaat wanneer twee gesynchroniseerde datacenters tegelijkertijd uitvallen, waardoor de continuïteit van de diensten en de integriteit van transacties in gevaar komt. 
Dit scenario is relevant bij grootschalige regionale incidenten zoals stroomuitval, natuurrampen of gecoördineerde cyberaanvallen.
Het toevoegen van een aanvullende derde site zorgt voor uitdagingen omdat door de afstand geen minimale vertragingstijd kan worden gegarandeerd.
Om dit risico te minimaliseren is het noodzakelijk een extra, geografisch gescheiden datacentrum toe te voegen, waarop alle transacties asynchroon kunnen worden gerepliceerd \autocite{snedaker2013business}.
Echter betekent een derde aanvullende site niet dat er geen invloed is op de vertragingstijd.
Hierdoor is een onderzoek naar verschillende oplossingen met betrekking tot een derde aanvullende site bij een oplossing van reeds twee gesynchroniseerde datacenters nuttig om een overzicht te krijgen van de invloed op Recovery Point Objective (RPO) en Recovery Time Objective (RTO).

\paragraph{Deelvragen probleemdomein}
\begin{itemize}
    \item Welke vormen van synchrone asynchrone DB2-logreplicatie bestaan binnen DB2 voor Z/OS en welke garanties bieden deze inzake transactionele consistentie?
    \item Welke technische beperkingen (vertragingstijd, afstand, commit-gedrag) verhinderen het gebruik van synchrone replicatie naar een derde geografisch afgelegen site?
    \item Welke faalscenario's kunnen optreden in een DB2-architectuur met meerdere sites en asynchrone replicatie en welke DB2-componenten zijn daarbij betrokken?
    \item Welke impact hebben deze faalscenario's op dataconsistentie, RPO en RTO\@?
\end{itemize}

\section{\IfLanguageName{dutch}{Onderzoeksvraag}{Research question}}%
\label{sec:onderzoeksvraag}

Deze bachelorproef analyseert de risico's bij IBM DB2 transactie-logreplicatie van kritieke bancaire mainframe-systemen naar geodistribueerde disaster recovery-omgevingen, met focus op RPO en RTO\@.

\section{\IfLanguageName{dutch}{Onderzoeksdoelstelling}{Research objective}}%
\label{sec:onderzoeksdoelstelling}

Het beoogd resultaat van de bachelorproef is een overzicht van verschillende oplossingen voor financiële instellingen met reeds twee gesynchroniseerde datacenters waarbij per oplossing RPO en RTO vermeld worden.
Bovendien zal per oplossing ook de vertragingstijd in kaart worden gebracht.

\section{\IfLanguageName{dutch}{Opzet van deze bachelorproef}{Structure of this bachelor thesis}}%
\label{sec:opzet-bachelorproef}

% Het is gebruikelijk aan het einde van de inleiding een overzicht te
% geven van de opbouw van de rest van de tekst. Deze sectie bevat al een aanzet
% die je kan aanvullen/aanpassen in functie van je eigen tekst.

De rest van deze bachelorproef is als volgt opgebouwd:

In Hoofdstuk~\ref{ch:literatuurstudie} wordt een overzicht gegeven van de stand van zaken binnen het onderzoeksdomein, op basis van een literatuurstudie.

In Hoofdstuk~\ref{ch:methodologie} wordt de methodologie toegelicht en worden de gebruikte onderzoekstechnieken besproken om een antwoord te kunnen formuleren op de onderzoeksvragen.

% TODO: Vul hier aan voor je eigen hoofstukken, één of twee zinnen per hoofdstuk

In Hoofdstuk~\ref{ch:conclusie}, tenslotte, wordt de conclusie gegeven en een antwoord geformuleerd op de onderzoeksvragen. Daarbij wordt ook een aanzet gegeven voor toekomstig onderzoek binnen dit domein.