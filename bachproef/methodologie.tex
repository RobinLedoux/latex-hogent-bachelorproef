\chapter{\IfLanguageName{dutch}{Methodologie}{Methodology}}%
\label{ch:methodologie}

Dit onderzoek werd uitgevoerd aan de hand van een technisch-analytische onderzoeksopzet bestaande uit vier hoofdfasen: een literatuurstudie, een architecturale analyse, een risicoanalyse en een evaluatiefase. 
De gekozen aanpak laat toe om zowel de theoretische fundamenten als de praktische implicaties van log-gebaseerde replicatie in DB2 voor Z/OS systematisch te onderzoeken.

In de eerste fase werd een technische literatuurstudie uitgevoerd op basis van academische publicaties en officiële documentatie. 
Hierbij werden verschillende bestaande replicatietechnieken binnen DB2 voor Z/OS geanalyseerd met bijzondere aandacht voor transactionele consistentie, herstelmechanismen en de implicaties voor RPO en RTO\@.

De studie omvatte documentatie over DB2 voor Z/OS, GDPS Global Mirror en Metro Mirror, logging in DB2 voor Z/OS aangevuld met academische literatuur over gedistribueerde systemen en replicatiemodellen. Deze fase resulteerde in een gestructureerd overzicht van replicatietechnieken en hun eigenschappen, samengevat in vergelijkende tabellen.

\section{\IfLanguageName{dutch}{Architecturale analyse}{Architectural analysis}}%
\label{sec:methodologie-architecturale-analyse}

In de tweede fase werd een vergelijkende architecturale analyse uitgevoerd van verschillende mainframe-opstellingen. Hierbij werd een configuratie met twee gesynchroniseerde sites vergeleken met een uitgebreide architectuur waarin een aanvullende asynchrone derde site werd toegevoegd.

Voor elke architectuur werden volgende zaken onderzocht.
\begin{itemize}
    \item Waarborging van dataconsistentie
    \item Activatie herstelprocedures bij uitval
    \item Implicaties voor RPO en RTO\@
\end{itemize}

De analyse resulteerde in architectuurdiagrammen en vergelijkende tabellen waarin consistentie-eigenschappen en herstelmechanismen per opstelling werden weergegeven.

\section{\IfLanguageName{dutch}{Risicoanalyse en faalscenario's}{Risk analysis and fail scenarios}}%
\label{sec:methodologie-risicoanalyse}

In de derde fase werd een systematische risicoanalyse uitgevoerd. 
Mogelijke faalscenario's binnen een DB2 voor Z/OS omgeving werden geïdentificeerd en gekoppeld aan betrokken componenten zoals log streams, coupling facilities en replicatiemechanismen.

Specifieke aandacht ging naar risico's geïntroduceerd door asynchrone replicatie, waaronder volgende zaken.
\begin{itemize}
    \item Replicatievertraging van commits
    \item Incomplete Logical Units of Work (LUW's)
    \item inconsistenties met betrekking tot de volgorde in logreplicatie
    \item Implicaties van manuele herstelprocedures
\end{itemize}

Voor elk scenario werd de impact op RPO, RTO en transactionele consistentie analytisch beoordeeld.

\section{\IfLanguageName{dutch}{Evaluatie en mitigatiestrategieën}{Evaluation and mitigitation strategies}}%
\label{sec:methodologie-evaluatie-mitigatie}

In de laatste fase werd een evaluatie uitgevoerd op basis van een risicomatrix waarin waarschijnlijkheid en impact werden gecombineerd. 
Hierbij werd gebruikgemaakt van een kwalitatieve risicoformule (risico = waarschijnlijkheid x impact).
Vervolgens werden mogelijke mitigatiestrategieën onderzocht, waaronder controlled log catch-up binnen DB2 voor Z/OS\@. 
Hierbij werd geanalyseerd onder welke voorwaarden deze techniek consistente herstelpunten kan garanderen en welke beperkingen dit oplegt aan de hersteltijd.

De resultaten van deze fase werden samengevat in een aanbevelingstabel waarin per architectuur de resterende risico's en de implicaties voor RPO, RTO en dataconsistentie werden weergegeven.