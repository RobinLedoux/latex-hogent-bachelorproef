\chapter{\IfLanguageName{dutch}{Literatuurstudie}{Literature study}}%
\label{ch:literatuurstudie}

% Tip: Begin elk hoofdstuk met een paragraaf inleiding die beschrijft hoe
% dit hoofdstuk past binnen het geheel van de bachelorproef. Geef in het
% bijzonder aan wat de link is met het vorige en volgende hoofdstuk.

% Pas na deze inleidende paragraaf komt de eerste sectiehoofding.

\section{\IfLanguageName{dutch}{Operationele veerkracht in financiële mainframe omgevingen}{Operational resilience in financial mainframe environments}}%
\label{sec:operationele-veerkracht}

In dit hoofdstuk wordt besproken wat operationele veerkracht inhoudt, doormiddel van bestaande literatuur hierover.
Er wordt ook besproken wat disaster recovery inhoudt in financiële mainframe-omgevingen.
Het concept van operationele veerkracht wordt eerst gekaderd in de context van financiële instellingen.
Hierna worden Recovery Point Objective (RPO) en Recovery Time Objective (RTO) gedefinieerd en de noodzaak van deze meetbare parameters uitgelegd.

Er is een nood voor cyberweerbaarheid bij financiële instellingen die niet enkel dekking biedt bij het voorkomen van cyberaanvallen, maar ook bij het herstellen wanneer er zich reeds een aanval heeft voorgedaan \autocite{dupont2019cyber}.
Verder stelt \textcite{dupont2019cyber} dat moderne organisaties in het algemeen, maar specifiek financiële instellingen gevoelig zijn aan het watervaleffect dat technische storingen, menselijke fouten en natuurrampen veroorzaken.
Bovendien worden de digitale middelen van financiële instellingen aangevallen door cybercriminelen, hackers vanuit regeringen, hacktivisten en ontevreden werknemers die trachten de computersystemen te infiltreren en gevoelige informatie te stelen.
Daaropvolgend helpt de toegankelijkheid van krachtige malware tools criminelen om aanvallen te realiseren waarbij gelimiteerde technische kennis voldoende is.
Operationele veerkracht verwijst naar het vermogen van organisaties om deze aanvallen te weerstaan, erop te reageren en schade voor klanten en de reputatie van het bedrijf te minimaliseren \autocite{essuman2020operational}.
Strengere regelgevingen, zoals DORA vanuit de Europese Unie, tonen overigens de nood aan operationele veerkracht bij financiële instellingen aan \autocite{clausmeier2023regulation}.
Financiële instellingen zijn in hoge mate afhankelijk van centrale transactieverwerkende systemen die vaak draaien op mainframeplatformen zoals Z/OS\@.
Zo gebruiken bedrijven, waaronder financiële instellingen, databanksystemen zoals DB2 voor Z/OS om hun bedrijfskritieke transacties te verwerken met hoge vereisten met betrekking tot beschikbaarheid, integriteit en consistentie \autocite{hernandez_2009_how}.
Operationele veerkracht verwijst in deze context niet enkel naar preventieve beveiligingsmaatregelen, maar ook robuuste herstelarchitecturen en geografisch gespreide replicatiemechanismen \autocite{bourbonnais2015}.
Om het herstellingsvermogen meetbaar te maken, worden doorgaans doelstellingen vastgelegd in de vorm van Recovery Time Objective en Recovery Point Objective \autocite{mendoncca2020evaluating}.
Bij een herstellingsplan is het voor financiële bedrijven noodzakelijk om meetbare criteria te hebben waaraan kan worden voldaan zodat kan worden bepaald binnen welke grenzen een organisatie een storing kan tolereren.
Hierin spelen Recovery Point Objective en Recovery Time Objective een cruciale rol \autocite{zgureanu2022role}.
Recovery Point Objective definieert de maximale hoeveelheid dataverlies, uitgedrukt in tijd, die aanvaardbaar is na een incident. 
Concreet geeft RPO aan hoe ver in de tijd een systeem mag worden teruggezet ten opzichte van het moment van verstoring zonder onaanvaardbare impact voor de organisatie \autocite{international2019security}. 
In replicatie-architecturen wordt de RPO rechtstreeks beïnvloed door de frequentie en mate van synchronisatie tussen primaire en secundaire systemen \autocite{swanson2010contingency}.
Recovery Time Objective verwijst naar de maximale tijdsduur waarbinnen een systeem of dienst opnieuw operationeel beschikbaar moet zijn na een storing. 
Deze parameter wordt beïnvloed door factoren zoals failover-mechanismen, herstelprocedures en de tijd nodig om transactielogs toe te passen en consistente toestanden te herstellen.
De gekozen replicatiestrategie en de complexiteit van de herstelprocedure hebben daarbij een directe impact op de uiteindelijke RTO\@.

\section{\IfLanguageName{dutch}{Replicatie in mainframe-omgevingen}{Replication in mainframe environments}}%
\label{sec:mainframe-replicatie}

Dit hoofdstuk bespreekt de verschillende manieren van data replicatie gebruikt binnen mainframe.
Er wordt gekaderd wat replicatie inhoudt en hoe dit op technisch vlak te werk gaat en welke toepassingen hiervoor worden gebruikt op mainframes met het Z/OS besturingsssysteem.

Replicatie houdt in dat data gekopiëerd wordt naar een secundaire site. 
Het doel hiervan is om redundantie te krijgen wanneer een primaire site uitvalt.
In mainframe is het mogelijk om op opslagniveau en op log-level te repliceren \autocite{bourbonnais2015}.
Op opslagniveau biedt IBM verscheidene oplossingen om redundantie te creëren binnen een mainframe met Z/OS\@.
IBM GDPS Metro Mirror is een replicatietechnologie die een synchrone remote-copy van gegevens verzorgt tussen een primaire en een secundaire site \autocite{gomes_2025_redbooks}.
Bij GDPS Metro Mirror wordt elke schrijfbewerking pas bevestigd aan de host (of het primaire systeem) als het ook op het secundaire systeem is geschreven.
Hierdoor is de data op het volume van het secundaire systeem identiek aan het volume van het primaire systeem op het moment dat het schrijven voltooid is.
Dit maakt Metro Mirror geschikt voor scenario's met nauwe synchronisatie en minimale data-divergentie binnen korte afstandsbereiken omdat de latentie tussen sites bij synchrone replicatie de responstijd van Input/Output (I/O) beïnvloedt.
Metro Mirror kan daarom worden gebruikt voor hoge beschikbaarheid en herstel na een ramp op kortere afstanden, vaak binnen een enkel geografisch gebied \autocite{ibm_2017_metro}.
Naast GDPS Metro Mirror biedt IBM ook GDPS Global Mirror aan, een asynchrone data replicatietechnologie die gegevens tussen een primaire en een externe site repliceert zonder de schrijfbewerking op de primaire site te blokkeren totdat de secundaire site deze heeft bewaard.
Dit betekent dat de primaire schrijfoperatie aan de host bevestigd wordt zodra die lokaal is voltooid, zonder te wachten op replicatie naar de secundaire site \autocite{ibm_2021_global}.
De replicatie gebeurt steeds na de schrijfoperatie op het primaire systeem waardoor er een natuurlijke vertraging kan ontstaan tussen de primaire en de secundaire kopie.
Global Mirror kan worden gebruikt voor lange afstanden (tot duizenden kilometers) omdat het repliceren geen invloed heeft op de host, want het primaire systeem wacht niet tot het repliceren voltooid is.
Met Global Mirror is het mogelijk om consistente punten (consistency groups) naar de externe externe site te verzenden, zodat een coherent herstelpunt ontstaat bij falen van het systeem.
GDPS Global Mirror maakt gebruik van een systeem met meesters (masters) en ondergeschikten (subordinates) waarbij een Global Mirror sessie maar één master mag hebben.
De master controleert het maken van consistency groups binnen een Global Mirror sessie en stuurt commando's naar de ondergeschikte opslagsystemen van de sessie. 

\section{\IfLanguageName{dutch}{Log-gebaseerde replicatie in DB2-omgevingen}{Log based replication in DB2 environments}}%
\label{sec:mainframe-replicatie}